\documentclass{article}
\usepackage[utf8]{inputenc}
\usepackage[margin=0.5in]{geometry}
\usepackage[portuguese]{babel}

\title{Sinais e Sistemas - Trabalho 1 - Grupo 2}
\author{
    Leonardo Soares da Costa Tanaka \\
    Matheus Henrique Sant Anna Cardoso \\
    Theo Rudra Macedo e Silva
}

\date{Setembro de 2022}

\begin{document}

\maketitle

1.) Para o sinal abaixo, contínuo por partes e definido para $t \in [-5\;5]$: \textbf{a} esboçar gráfico, \textbf{b} encotrar uma expressão analítica usando sinais singulares, \textbf{c} escrever um programa que rode em Octave/MatLab para plotar o gráfico. Nos dados a seguir, as expressões entre vírgulas se referem, na ordem de apresentação, aos valores do sinal nos intervalos $I_{1} = [-5\;-3], I_{2} = [-3\;-1], I_{3} = [-1\;1], I_{4} = [1\;3], I_{5} = [3\;5]$.
\textbf{G2}: $x(t) = 3, 3t + 6, -3t^3, 3t - 6, -t^2 + 5t - 3$

\vspace{\baselineskip}

2.) Plotar o gráfico dos sinais a seguir, com escalas adequadas e usando os valores numéricos desejados para os eventuais parâmetros. Dizer se estes sinais são periódicos e, em caso afirmativo quais os seus períodos fundamentais.
(a) $x(t) = sen(\pi t) + cos(2 \pi t) / 2 + sen(3 \pi t) / 3 + cos(4 \pi t) / 4$,
(b) $x(t) = sen(\omega t)cos(50\omega t)$, (c) $x(t) = sen(\omega t^2)$, (d) $x(t) = sen(\omega_{1}sen(\omega_{2}t)t)$

\vspace{\baselineskip}

3.) Um sinal periódico com período fundamental $T_{0} = 4 $ é descrito por \textbf{G2}: $x(t) = 1 - e^{|t|}$ para $-T_{0}/2 \leq t < T_{0} / 2$
(a) Esboce o seu gráfico;
(b) calcule analiticamente sua potência total $P$;
(c) calcule $X_{0}$ usando $k = 0$ na fórmula geral de $X_{k}$;
(d) calcule analiticamente os coeficientes $X_{k}$ e verifique se a expressão obtida leva a $X_{0}$ sem indeterminações;
(e) esboce os espectros de módulo e fase;
(f) para $k = 0, 1, 2\;e\;3$, calcule a potência acumulada $P_{k}^{a}$ contida nos harmônicos de $0\;a\;k$;
(g) para $k = 0, 1, 2\;e\;3$, calcule a potência relativa $P_{k}^{a}/P$;
(h) quantos harmônicos são necessários para uma aproximação reter $90.00\%$ da potência?

\vspace{\baselineskip}

4.) O grupo $i$ trabalhará com o sinal periódico $x(t)$ usado pelo grupo $i + 1$ na questão 1 (ao grupo 7: sinal 1). As aproximações numéricas para Octave/MatLab vistas, podem e devem ser utilizadas.
(a) Traçar gráfico;
(b) encontrar potência total $P$;
(c) calcular os $X_{k}$ para $k \in [-10\;10]$;
(d) traçar os espectos de magnitude, fase e potência;
(e) estimar quantos harmônicos são necessários para reter $90.00\%$ da potência;
(f) calcular os coeficientes $a_{k}\;e\;b_{k}$ correspondentes;
(g) traçar, num mesmo gráfico, $x(t)$ e as aproximações.

\vspace{\baselineskip}

5.) Na escala de tempo {\tt t=0:1/2000:5}, considere um sinal de áudio simples $x_{b}(t) = sen(2 \pi f_{0}t)$ ou $x_{b}(t) = cos(2 \pi f_{0}t)$ com frequência \textbf{G2:} $f_{0} = 132Hz$. Ouça este som usando o comando {\tt sound(xb)} no Octave; o resultado é, provavelmente, desagradável pois se trata de uma frequência pura e a sensação é seca, metálica. Para melhorar o \textbf{timbre} do som é preciso colocar mais harmônicos. Crie, na mesma escala de tempo, com a mesma frequência fundamental $f_{0}$, com parâmetros a seu critério e usando até o harmônico $k = 6\;(6f_{0}Hz)$ os sinais a seguir. Ouça cada um deles e compare a qualidade do timbre.
(a) Uma onda quadrada $x_{q}(t)$;
(b) uma onda triangular $x_{t}(t)$;
(c) um seno semi-retificado sem o nível DC $x_{s}(t)$;
(d) Opcional: adicionando senos e co-senos harmônicos a seu critério, imagine-se projetando um sintetizador de som e crie um sinal periódico $x(t)$ com frequência fundamental $f_{0}$ e um timbre agradável.


\end{document}
