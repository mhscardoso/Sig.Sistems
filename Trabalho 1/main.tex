\documentclass{article}
\usepackage[utf8]{inputenc}
\usepackage[margin=0.5in]{geometry}

\title{Sinais e Sistemas - Trabalho 1}
\author{
    Leonardo Soares da Costa Tanaka \\
    Matheus Henrique Sant Anna Cardoso \\
    Theo Rudra Macedo e Silva
}

\date{Setembro 2022}

\begin{document}

\maketitle

1.) Para o sinal abaixo, contínuo por partes e definido para $t \in [-5\;5]$: \textbf{a} esboçar gráfico, \textbf{b} encotrar uma expressão analítica usando sinais singulares, \textbf{c} escrever um programa que rode em Octave/MatLab para plotar o gráfico. Nos dados a seguir, as expressões entre vírgulas se referem, na ordem de apresentação, aos valores do sinal nos intervalos $I_{1} = [-5\;-3], I_{2} = [-3\;-1], I_{3} = [-1\;1], I_{4} = [1\;3], I_{5} = [3\;5]$.\\
\textbf{G1}: $x(t) = 3, -3t - 6, 3t^3, -3t + 6, t^2 -5t + 3$;\;
\textbf{G2}: $x(t) = 3, 3t + 6, -3t^3, 3t - 6, -t^2 + 5t - 3$;\;
\textbf{G3}: $x(t) = -t -5, -2, -t^3 + 3t, 2, -t + 5$;\;
\textbf{G4}: $x(t) = -2, t + 1, (t^3 - t) / 2, t - 1, 2$;\;
\textbf{G5}: $x(t) = 2t + 8, -2t - 4, -(2t^3 - t^2 - 4t + 3) / 2,0, -t + 3$;\;
\textbf{G6}: $x(t) = -t-5- sen(2\pi (t - 5))/(2\pi),-2,2t + sen(2\pi t)/\pi,2, -t + 5 + sen(2\pi (t - 3)) / (2 \pi )$;\;
\textbf{G7}: $x(t) = 3 + t sen(\pi (t + 5)),3 + t + sen(2\pi (t + 3)), 3 - t^2 - sen(3\pi (t + 1)), 3 - t - sen(2\pi (t - 1)),3 - t - sen(\pi (t - 3))$;\;



\end{document}
