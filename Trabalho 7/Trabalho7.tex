\documentclass[10pt]{article}

% Pacotes extras necessários
\usepackage{amsmath}
\usepackage[lmargin=0.5in, rmargin=0.5in, tmargin=0.5in, bmargin=0.5in, includehead, includefoot]{geometry}
\usepackage{amsfonts}
\usepackage[utf8]{inputenc}
\usepackage[portuguese]{babel}
\usepackage{graphicx}
\usepackage{fancyhdr}
\usepackage{setspace}

\graphicspath{ {./images/} }

% Sets para outras partes
\setlength{\parindent}{0pt}
\setstretch{1.5}
\DeclareMathOperator{\sen}{sen}
\DeclareMathOperator{\sinc}{sinc}

%% Facilidades
%% -- Laplace
\newcommand{\Lap}[1]{\mathcal{L}\left\{#1\right\}}

%% -- Fourier
\newcommand{\Fou}[1]{\mathcal{F}\left\{#1\right\}}

%% -- Transformada Z
\newcommand{\Z}[1]{\mathcal{Z}\left\{#1\right\}}

%% -- Negrito em matemáticas
\newcommand{\bm}[1]{\boldsymbol{#1}}


% ------- Estilo do trabalho -------- %
\fancypagestyle{capa}{
    \fancyhf{}
    \renewcommand\headrulewidth{0pt}
    \fancyfoot[C]{
        Rio de Janeiro\\
        2022
    }
}

\pagestyle{fancy}
\fancyhead{}
\fancyhead[L]{\thepage}
\fancyfoot{}
% ----------------------------------- %

% Dados do Grupo
\title{Sinais e Sistemas - Trabalho 7 - Avaliação 11}
\author{
    \textbf{Grupo 2}\\
    Leonardo Soares da Costa Tanaka\\
    Matheus Henrique Sant Anna Cardoso\\
    Theo Rudra Macedo e Silva\\
    Vinícius Quintanilha Porto Gomes
}
\date{}

\begin{document}
\maketitle
\thispagestyle{capa}
\newpage

\textbf{1.)} Para o sinal abaixo:

\textbf{G2: } $x = [30 \ \ 20 \ \ 12 \ \ 6 \ \ 2 \ \ 0 \ \ 0 \ \ 2 \ \ 6 \ \ 12 \ \ 20 \ \ 30]$

(a) compute a primeira tendência e a primeira flutuação, por Haar;

(b) determine a porcentagem de compactação, ou seja, a energia do sub de tendência dividida pela energia total;

(c) compute uma aproximação $\tilde{x}$ anulando as primeiras flutuações $(d_i^1 = 0)$;

(d) ache a inversa do sinal resultante;

(e) avalie a qualidade da aproximação, pela porcentagem de energia presente.

(f) é possível, desprezando elementos de pequenos módulos, conseguir uma aproximação que retém $99.99\%$ da energia?

(g) reperir (f) para o nível 2 de Haar.

\vspace{\baselineskip}

\textbf{2.)} Os pulsos a seguir são pares e nulos para $|t| > \Delta:$

$p_{\Delta}(t)$ é o plano com $p_{\Delta}(t) = \Delta$ para $|t| \leq \Delta$,

$r_{\Delta}$ é triangular com $r_{\Delta}(-\Delta) = r_{\Delta}(\Delta) = 0$ e $r_{\Delta}(0) = \pi \Delta / 2$ e

$c_{\Delta}$ é uma semicircunferência com $c_{\Delta}(-\Delta) = c_{\Delta}(\Delta) = 0$ e $c_{\Delta}(0) = \Delta$.

\vspace{\baselineskip}

(a) Esboçar os gráficos para os três pulsos e para $x = p_4(t) + r_2(t - 2) - c_2(t + 2)$;

(b) traçar o espectro de magnitude para $x(t)$, via FFT, determinando $T_0$ e $f_a$ por tentativa e erros;

(c) com a mesma janela, e o número de amostras aproximado para uma potência de 2, obter Haar 1;

(d) obter a Haar inversa do sinal truncado para reter $90.00\%$ da energia e plotar no mesmo gráfico;

(e) idem (d) para reter $99.99\%$ da energia e plotar no mesmo gráfico.


\vspace{\baselineskip}


\textbf{3.)} Para o sinal a seguir:
\[x(t) = 8\sinc(4t) - 2\sinc(2t)\]

(a) plote o gráfico;

(b) encontre, justificando, a largura $T_0$ de uma janela de observação centrada na origem;

(c) idem período de amostragem $\Delta t$ segura;

(d) encontre o número de pontos $N = 2^p$;

(e) que limiar deve ser usado para reter $99.99\%$ da energia?;

(f) que taxa de compressão isto produz? (Fazer para os níveis 1, 2, 3 e 10 de Haar, uma Daub qualquer de sua escolha e uma Coif qualquer de sua escolha);

(g) comparar os resultados.



\end{document}