\documentclass{article}

% Pacotes extras necessários
\usepackage{amsmath}
\usepackage[lmargin=0.7in, rmargin=0.7in, tmargin=0.5in, bmargin=0.5in, includehead, includefoot]{geometry}
\usepackage{amsfonts}
\usepackage[utf8]{inputenc}
\usepackage[portuguese]{babel}
\usepackage{fancyhdr}

% Sets para outras partes
\setlength{\parindent}{0pt}
\DeclareMathOperator{\sen}{sen}

% ------- Estilo do trabalho -------- %
\fancypagestyle{capa}{
    \fancyhf{}
    \renewcommand\headrulewidth{0pt}
    \fancyfoot[C]{
        Rio de Janeiro\\
        2022
    }
}

\pagestyle{fancy}
\fancyhead{}
\fancyhead[L]{\thepage}
\fancyfoot{}
% ----------------------------------- %

% Dados do Grupo
\title{Sinais e Sistemas - Trabalho 3 - Avaliação 5}
\author{
    \textbf{Grupo 2}\\
    Leonardo Soares da Costa Tanaka\\
    Matheus Henrique Sant Anna Cardoso\\
    Theo Rudra Macedo e Silva
}
\date{}


\begin{document}

% Capa
\maketitle
\thispagestyle{capa}

\newpage

\textbf{1.)} Um SLIT é modelado por $\tau \dot{y}(t) + y(t) = u(t)$ com $y(0^{-}) = \alpha$.
\textbf{G2:} $\tau = 3, \alpha = -2$

(a) Calcular a resposta ao degrau unitário e esboçar o seu gráfico;

Para calcular a resposta, trataremos já com os dados, sendo a EDO:
\[3\dot{y}(t) + y(t) = 1(t)\,\,\,\,,\,\,\,\,y(0^{-}) = -2\]
Pela propriedade da derivação, sabemos que
\[\dot{y}(t) = sY(s) - y(0^{-})\]
Então
\begin{align*}
    \mathcal{L}\{3\dot{y}(t) + y(t)\} &= \mathcal{L}\{u(t)\}\\
    3(sY(s) - y(0^-)) + Y(s) &= U(s)\\
    Y(s)(3s + 1) + 6 &= U(s)
\end{align*}

Como $u(t) = 1(t)$, sabemos que $U(s) = \frac{1}{s}$, assim

\begin{align*}
    Y(s)(3s + 1) + 6 = \frac{1}{s}\\
    Y(s) = \frac{1}{3s + 1} \cdot \frac{1}{s} - \frac{6}{3s + 1}
\end{align*}

Separando em frações parciais, temos:

\begin{align*}
    Y(s) &= \frac{1}{s} - \frac{3}{3s + 1} - \frac{6}{3s + 1}\\
    Y(s) &= \frac{1}{s} - \frac{9}{3s + 1}\\
    Y(s) &= \frac{1}{s} - \frac{3}{s + 1/3}
\end{align*}

Agora, podemos descobrir $y(t)$.

\begin{align*}
    \mathcal{L}^{-1} \biggl\{\frac{1}{s}\biggr\} &= 1(t) & \mathcal{L}^{-1}\biggl\{\frac{3}{s + 1/3}\biggr\} &= 3e^{-\frac{1}{3}t}1(t)
\end{align*}

\begin{align*}
    \mathcal{L}^{-1} \left\{Y(s)\right\} &= 1(t) - 3e^{-\frac{1}{3}t}1(t)
\end{align*}

Finalmente

\[y(t) = 1(t) - 3e^{-\frac{t}{3}}1(t)\]


(b) calcular a resposta à rampa unitária e esboçar o seu gráfico;

Para calcular a resposta à rampa, tomemos a seguinte EDO:

\[3\dot{y}(t) + y(t) = t1(t) \,\,\,\,,\,\,\,\,y(0^-) = -2\]

Da mesma forma como no item (a), podemos utilizar a propriedade da derivação e fazer em ambos os lados, a transformada de Laplace.

\begin{align*}
    \mathcal{L}\left\{\dot{y}(t)\right\} &= \left(sY(s) - y(0^-)\right) & \mathcal{L}\left\{t1(t)\right\} = U(s) &= \frac{1}{s^2}\\
\end{align*}

\begin{align*}
    3(Y(s) - y(0^-)) + Y(s) = U(s)\\
    Y(s)(3s + 1) + 6 = \frac{1}{s^2}\\
    Y(s) = \frac{1}{3s + 1}\cdot\frac{1}{s^2} - \frac{6}{3s + 1}
\end{align*}

Separando em frações parciais, teremos

\begin{align*}
    Y(s) &= \frac{9}{3s + 1} - \frac{3s - 1}{s^2} - \frac{6}{3s + 1}\\
    Y(s) &= \frac{3}{3s + 1} - \frac{3s - 1}{s^2}\\
    Y(s) &= \frac{1}{s + 1/3} - \frac{3}{s} + \frac{1}{s^2}
\end{align*}

Agora, podemos calcular a inversa da transformada de Laplace.

\begin{align*}
    \mathcal{L}^{-1} \left\{\frac{1}{s + 1/3}\right\} &= e^{-\frac{t}{3}}1(t) & \mathcal{L}^{-1} \left\{\frac{3}{s}\right\} &= 3\cdot1(t) & \mathcal{L}^{-1} \left\{\frac{1}{s^2}\right\} &= t1(t)\\
\end{align*}
\begin{align*}
    \mathcal{L}^{-1} \left\{Y(s)\right\} &= e^{-\frac{t}{3}}1(t) - 3\cdot1(t) + t1(t)
\end{align*}
Finalmente
\[y(t) = e^{-\frac{t}{3}}1(t) - 3\cdot1(t) + t1(t)\]

(c) calcular a resposta ao seno $u(t) = \sen(\omega t)$ para $\alpha = 0,\, \omega = 1/(4\tau)$ e esboçar o seu gráfico;

Agora, temos para resolver a seguinte EDO:

\[ 3\dot{y}(t) + y(t) = \sen\left(\frac{t}{12}\right)1(t) \,\,\,\,,\,\,\,\, y(0^-) = 0\]

Utilizaremos Laplace, para resolver, a saber

\begin{align*}
    \mathcal{L}\left\{\dot{y}(t)\right\} &= sY(s) & \mathcal{L}\left\{\sen \left(\frac{t}{12}1(t)\right) \right\} &= \frac{1/12}{s^2 + 1 / 144} = \frac{12}{144s^2 + 1}
\end{align*}

Utilizando, novamente, as técnicas empregadas nos itens anteriores, teremos

\begin{align*}
    3(sY(s)) + Y(s) = \frac{12}{144s^2 + 1}\\
    Y(s)(3s + 1) = \frac{12}{144s^2 + 1}\\
    Y(s) = \frac{1}{3s + 1} \cdot \frac{12}{144s^2 + 1}\\
    Y(s) = \frac{r_1}{3s + 1} + \frac{r_2}{144s^2 + 1}
\end{align*}

Vamos, por tentativa e erro, calcular as constantes $r_1 \text{ e } r_2$.

Perceba que, fazendo $r_1 \cdot (144s^2 + 1)$ teremos um termo com $s^2$ mais uma constante. O mesmo deve ser para $r_2 \cdot (3s + 1)$. Para isso, podemos fazer com que o segundo seja o produto da soma pela diferença, obtendo um termo ao quadrado e uma constante.

Fazemos, então $r_2 = (3s - 1)$, obtendo, naquele segundo produto, o seguinte: $r_2 \cdot (3s + 1) = (3s - 1) (3s + 1) = 9s^2 - 1$.

No primeiro produto ($r_1 \cdot (144s^2 + 1)$), já temos um fator grande suficiente para o termo quadrático, podemos corrigir com o segundo produto, multiplicando por 16.

No final, ficamos com a expressão

\[\frac{1}{3s + 1} - \frac{16(3s - 1)}{144s^2 + 1}\]

Ficamos, porém, com uma constante no numerador igual a ($144s^2 + 1 - 16(9s^2 - 1) = 1 + 16 = 17$) diferente da expressão original ($12$). Sendo assim, podemos multiplicar a expressão toda por $\frac{12}{17}$ para termos a expressão inicial na forma de somas parciais.

Então

\[Y(s) = \left(\frac{1}{3s + 1} - \frac{16(3s - 1)}{144s^2 + 1}\right) \cdot \frac{12}{17}\]

Assim

\[Y(s) \cdot \frac{17}{12} = \frac{1}{3s + 1} - \frac{48s - 16}{144s^2 + 1}\]

Fazendo mais alterações, teremos

\begin{align*}
    Y(s) \cdot \frac{17}{12} = \frac{1}{3s + 1} - \frac{48s}{144s^2 + 1} + \frac{16}{144s^2 + 1}\\
    Y(s) \cdot \frac{17}{12} = \frac{1/3}{s + 1/3} - \frac{1}{3}\cdot\frac{s}{s^2 + 1/144} + \frac{4}{3}\cdot\frac{1/12}{s^2 + 1/144}
\end{align*}

\begin{align*}
    \mathcal{L}^{-1} \left\{\frac{1}{s + 1/3}\right\} &= e^{-\frac{t}{3}}1(t) & \mathcal{L}^{-1} \left\{\frac{s}{s^2 + 1/144}\right\} &= \cos\left(\frac{t}{12}\right)1(t) & \mathcal{L}^{-1} \left\{\frac{1/12}{s^2 + 1/144}\right\} &= \sen\left(\frac{t}{12}\right)1(t)
\end{align*}

\begin{align*}
    \mathcal{L}^{-1} \left\{Y(s) \cdot \frac{17}{12}\right\} = \frac{e^{-\frac{t}{3}}1(t)}{3} - \frac{\cos\left(\frac{t}{12}\right)1(t)}{3} + \frac{4}{3}\cdot sen\left(\frac{t}{12}\right)1(t)\\
    y(t) \cdot \frac{17}{4} = e^{-\frac{t}{3}}1(t) - \cos\left(\frac{t}{12}\right)1(t) + 4sen\left(\frac{t}{12}\right)1(t)\\
\end{align*}

Finalmente

\[y(t) = 1(t)\cdot\frac{4e^{-\frac{t}{3}} - 4\cos\left(\frac{t}{12}\right) + 16sen\left(\frac{t}{12}\right)}{17}\]

(d) calcular a resposta ao seno $u(t) = \sen(\omega t)$ para $\alpha = 0,\, \omega = 4/\tau$ e esboçar o seu gráfico;

Agora, temos para resolver a seguinte EDO:

\[ 3\dot{y}(t) + y(t) = \sen\left(\frac{4t}{3}\right)1(t) \,\,\,\,,\,\,\,\, y(0^-) = 0\]

Utilizaremos Laplace, para resolver, a saber

\begin{align*}
    \mathcal{L}\left\{\dot{y}(t)\right\} &= sY(s) & \mathcal{L}\left\{\sen \left(\frac{4t}{3}1(t)\right) \right\} &= \frac{4/3}{s^2 + 16/9} = \frac{12}{9s^2 + 16}
\end{align*}

Utilizando, novamente, as técnicas empregadas nos itens anteriores, teremos

\begin{align*}
    3(sY(s)) + Y(s) = \frac{12}{9s^2 + 16}\\
    Y(s)(3s + 1) = \frac{12}{9s^2 + 16}\\
    Y(s) = \frac{1}{3s + 1} \cdot \frac{12}{9s^2 + 16}\\
    Y(s) = \frac{r_1}{3s + 1} + \frac{r_2}{9s^2 + 16}
\end{align*}

Vamos, por tentativa e erro, calcular as constantes $r_1 \text{ e } r_2$.

Perceba que, fazendo $r_1 \cdot (9s^2 + 16)$ teremos um termo com $s^2$ mais uma constante. O mesmo deve ser para $r_2 \cdot (3s + 1)$. Para isso, podemos fazer com que o segundo seja o produto da soma pela diferença, obtendo um termo ao quadrado e uma constante.

Fazemos, então $r_2 = (3s - 1)$, obtendo, naquele segundo produto, o seguinte: $r_2 \cdot (3s + 1) = (3s - 1) (3s + 1) = 9s^2 - 1$.

No primeiro produto ($r_1 \cdot (9s^2 + 16)$), já temos um fator multiplicado pelo termo quadrático, igual ao outro. Não tendo a necessidade de multiplicarmos por nada. Assim, ficamos com $r_1 = 1$.

No final, ficamos com a expressão

\[\frac{1}{3s + 1} - \frac{(3s - 1)}{9s^2 + 16}\]

Ficamos, porém, com uma constante no numerador igual a ($9s^2 + 16 - (9s^2 - 1) = 16 + 1 = 17$) diferente da expressão original ($12$). Sendo assim, podemos multiplicar a expressão toda por $\frac{12}{17}$ para termos a expressão inicial na forma de somas parciais.

Então

\[Y(s) = \left(\frac{1}{3s + 1} - \frac{3s - 1}{9s^2 + 16}\right) \cdot \frac{12}{17}\]

Assim

\[Y(s) \cdot \frac{17}{12} = \frac{1}{3s + 1} - \frac{3s - 1}{9s^2 + 16}\]

Fazendo mais alterações, teremos

\begin{align*}
    Y(s) \cdot \frac{17}{12} = \frac{1}{3s + 1} - \frac{3s}{9s^2 + 16} + \frac{1}{9s^2 + 16}\\
    Y(s) \cdot \frac{17}{12} = \frac{1/3}{s + 1/3} - \frac{1}{3}\cdot\frac{s}{s^2 + 16/9} + \frac{1}{12}\cdot\frac{4/3}{s^2 + 16/9}
\end{align*}

\begin{align*}
    \mathcal{L}^{-1} \left\{\frac{1}{s + 1/3}\right\} &= e^{-\frac{t}{3}}1(t) & \mathcal{L}^{-1} \left\{\frac{s}{s^2 + 1/144}\right\} &= \cos\left(\frac{t}{12}\right)1(t) & \mathcal{L}^{-1} \left\{\frac{1/12}{s^2 + 1/144}\right\} &= \sen\left(\frac{t}{12}\right)1(t)
\end{align*}

\begin{align*}
    \mathcal{L}^{-1} \left\{Y(s) \cdot \frac{17}{12}\right\} = \frac{e^{-\frac{t}{3}}1(t)}{3} - \frac{\cos\left(\frac{4t}{3}\right)1(t)}{3} + \frac{1}{12}\cdot sen\left(\frac{4t}{3}\right)1(t)\\
    y(t) \cdot 17 = 4e^{-\frac{t}{3}}1(t) - 4\cos\left(\frac{4t}{3}\right)1(t) + sen\left(\frac{t}{12}\right)1(t)\\
\end{align*}

Finalmente

\[y(t) = 1(t)\cdot\frac{4e^{-\frac{t}{3}} - 4\cos\left(\frac{4t}{3}\right) + sen\left(\frac{4t}{3}\right)}{17}\]

(e) encontrar a entrada $u(t)$ para que $y(t) = \alpha\,\forall t \geq 0$;

Trabalhando com a seguinte EDO, teremos:

\[3\dot{y}(t) + y(t) = u(t)\,\,\,\,,\,\,\,\,y(0^-) = -2\]

Porém, temos que $y(t) = -2 \forall t \geq 0$. Assim, podemos definir a função $y(t) = -2$, pois respeitará o fato de $y(0^-) = -2$. Dessa forma, $\dot{y}(t) = 0$. Na equação, teremos:

\[3\cdot 0 + (-2) = u(t)\]
Finalmente
\[u(t) = -2\]

(f) encontrar a entrada $u(t)$ para que  $y(t) = 0\,\forall t > 0$.

Aqui, devemos tomar o cuidado para manter a condição de que $y(0^-) = -2$. Para isso, podemos definir que
\[y(t) = -2 \cdot 1(-t)\]
Tendo, por consequência que
\[\dot{y}(t) = -2\delta(-t) = -2\delta(t)\]
Acima, utilizamos a inversão na escala de tempo para chegar que $\dot{y}(t) = -2\delta(t)$.

Agora, podemos resolver a equação, que é dada por

\[3\dot{y}(t) + y(t) = u(t)\,\,\,\,,\,\,\,\,y(0^-) = -2\]
\[3(-2\delta(t)) + (-2\cdot 1(t)) = u(t)\]
Finalmente
\[u(t) = -6\delta(t) - 2\cdot 1(t)\]

\vspace{\baselineskip}


\textbf{2.)} Um SLIT relaxado é descrito por $\ddot{y}(t) + a_1\dot{y}(t) + a_0y(t) = b_1\dot{u}(t) + u(t)$.
\textbf{G2:} $a_1 = 30, a_0 = 3$

(a) Para $b_1 = 0$ calcular a resposta ao degrau unitário e esboçar o seu gráfico;

\vspace{\baselineskip}

\[\ddot{y}(t) + 30\dot{y}(t) + 3y(t) = u(t)\,\,\,\,,\,\,\,\,y(0^{-}) = 0\]

\[\mathcal{L} \left\{\ddot{y}(t) + 30\dot{y}(t) + 3y(t)\right\} = \mathcal{L} \left\{1(t)\right\}\]

\[ s^{2}Y(s) + 30sY(s) + 3Y(s) = \frac{1}{s} \]

\[ Y(s) = \frac{1}{s \cdot (s^{2} + 30s + 3)} = \frac{1}{s} \cdot \frac{1}{(s + 15 - \sqrt{222}) \cdot (s + 15 + \sqrt{222})} =\]

\[ = \frac{1}{s} \cdot \left(\frac{ \frac{\sqrt{222}}{444}}{s + 15 - \sqrt{222}} - \frac{\frac{\sqrt{222}}{444}}{s + 15 + \sqrt{222}}\right) \]

\[ y(t) = \int \mathcal{L}^{-1} \left\{\frac{\sqrt{222}}{444} \cdot \left[ \frac{1}{s + 15 - \sqrt{222}} - \frac{1}{s + 15 + \sqrt{222}} \right] \right\} = \]

\[ = \frac{\sqrt{222}}{444} \cdot \int (e^{(-15+\sqrt{222})t} - e^{(-15-\sqrt{222})t}) dt = \frac{\sqrt{222}}{444} \cdot \left\{ \frac{e^{(-15+\sqrt{222})t}}{-15+\sqrt{222}} + \frac{e^{(-15+\sqrt{222})t}}{15+\sqrt{222}} \right\}  \]

\vspace{\baselineskip}

(b) para $b_1 = 0$ calcular a resposta à rampa unitária e esboçar o seu gráfico;

\vspace{\baselineskip}

\[\ddot{y}(t) + 30\dot{y}(t) + 3y(t) = u(t)\,\,\,\,,\,\,\,\,y(0^{-}) = 0\]

\[\mathcal{L} \left\{\ddot{y}(t) + 30\dot{y}(t) + 3y(t)\right\} = \mathcal{L} \left\{t1(t)\right\}\]

\[ s^{2}Y(s) + 30sY(s) + 3Y(s) = \frac{1}{s^{2}} \]

\[ Y(s) = \frac{1}{s^{2} \cdot (s^{2} + 30s + 3)} = \frac{1}{s^{2}} \cdot \frac{1}{(s + 15 - \sqrt{222}) \cdot (s + 15 + \sqrt{222})} =\]

\[ = \frac{1}{s^{2}} \cdot \left(\frac{ \frac{\sqrt{222}}{444}}{s + 15 - \sqrt{222}} - \frac{\frac{\sqrt{222}}{444}}{s + 15 + \sqrt{222}}\right) \]

\[ y(t) = \int \int \mathcal{L}^{-1} \left\{\frac{\sqrt{222}}{444} \cdot \left[ \frac{1}{s + 15 - \sqrt{222}} - \frac{1}{s + 15 + \sqrt{222}} \right] \right\} = \]

\[ = \frac{\sqrt{222}}{444} \cdot \int \left\{ \frac{e^{(-15+\sqrt{222})t}}{-15+\sqrt{222}} + \frac{e^{(-15+\sqrt{222})t}}{15+\sqrt{222}} \right\} dt = \frac{\sqrt{222}}{444} \cdot \left\{ \frac{e^{(-15+\sqrt{222})t}}{(-15+\sqrt{222})^{2}} + \frac{e^{(-15+\sqrt{222})t}}{(15+\sqrt{222})^{2}} \right\} \]

\vspace{\baselineskip}

(c) para $b_1 = 1$ calcular a resposta ao degrau unitário e esboçar o seu gráfico;

\[\ddot{y}(t) + 30\dot{y}(t) + 3y(t) = \dot{u}(t) + u(t)\,\,\,\,,\,\,\,\,y(0^{-}) = 0\]

\[\mathcal{L} \left\{\ddot{y}(t) + 30\dot{y}(t) + 3y(t)\right\} = \mathcal{L} \left\{1(t) + t1(t)\right\}\]

\[ s^{2}Y(s) + 30sY(s) + 3Y(s) = 1 + \frac{1}{s} \]

\[Y(s) = \frac{1}{s^{2} + 30s + 3} + \frac{1}{s \cdot (s^{2} + 30s + 3)}\]

\[ Y(s) = \frac{1}{(s + 15 - \sqrt{222}) \cdot (s + 15 + \sqrt{222})} + \frac{1}{s} \cdot \frac{1}{(s + 15 - \sqrt{222}) \cdot (s + 15 + \sqrt{222})}\]

\[ = \frac{ \frac{\sqrt{222}}{444}}{s + 15 - \sqrt{222}} - \frac{\frac{\sqrt{222}}{444}}{s + 15 + \sqrt{222}} + \frac{1}{s} \cdot \left(\frac{ \frac{\sqrt{222}}{444}}{s + 15 - \sqrt{222}} - \frac{\frac{\sqrt{222}}{444}}{s + 15 + \sqrt{222}}\right) \]

\[ = \frac{\sqrt{222}}{444} \cdot \left\{ e^{(-15+\sqrt{222})t} - e^{(-15+\sqrt{222})t} + \frac{e^{(-15+\sqrt{222})t}}{-15+\sqrt{222}} + \frac{e^{(-15+\sqrt{222})t}}{15+\sqrt{222}} \right\} \]

\vspace{\baselineskip}

(d) para $b_1 = -1$ calcular a resposta ao degrau unitário e esboçar o seu gráfico;

\vspace{\baselineskip}

\[\ddot{y}(t) + 30\dot{y}(t) + 3y(t) = - \dot{u}(t) + u(t)\,\,\,\,,\,\,\,\,y(0^{-}) = 0\]

\[\mathcal{L} \left\{\ddot{y}(t) + 30\dot{y}(t) + 3y(t)\right\} = \mathcal{L} \left\{- 1(t) + t1(t)\right\}\]

\[ s^{2}Y(s) + 30sY(s) + 3Y(s) = - 1 + \frac{1}{s} \]

\[Y(s) = - \frac{1}{s^{2} + 30s + 3} + \frac{1}{s \cdot (s^{2} + 30s + 3)}\]

\[ Y(s) = \frac{1}{(s + 15 - \sqrt{222}) \cdot (s + 15 + \sqrt{222})} + \frac{1}{s} \cdot \frac{1}{(s + 15 - \sqrt{222}) \cdot (s + 15 + \sqrt{222})}\]

\[  = - \frac{ \frac{\sqrt{222}}{444}}{s + 15 - \sqrt{222}} + \frac{\frac{\sqrt{222}}{444}}{s + 15 + \sqrt{222}} + \frac{1}{s} \cdot \left(\frac{ \frac{\sqrt{222}}{444}}{s + 15 - \sqrt{222}} - \frac{\frac{\sqrt{222}}{444}}{s + 15 + \sqrt{222}}\right) \]

\[ = \frac{\sqrt{222}}{444} \cdot \left\{ - e^{(-15+\sqrt{222})t} + e^{(-15+\sqrt{222})t} + \frac{e^{(-15+\sqrt{222})t}}{-15+\sqrt{222}} + \frac{e^{(-15+\sqrt{222})t}}{15+\sqrt{222}} \right\} \]

\vspace{\baselineskip}

(e) traçar com precisão um único gráfico com as 3 respostas ao degrau;

(f) calcular a resposta a seno $u(t) = \sen(\omega t)$ para $b_1 = 0, \omega = 4a_0$ e esboçar o seu gráfico.


\vspace{\baselineskip}

\textbf{3.)} Um SLIT relaxado é descrito por uma função de transferência com numerador e denominador dados por $n(s) = K$ e $d(s) = (s + p_1)(s + p_2)(s + p_3)$.
\textbf{G2:} $p_1 = 1, p_2 = 3, p_3 = 4$

(a) Esboçar a resposta ao degrau unitário (escolha o valor de $K$);

(b) esboçar a resposta ao degrau unitário quando os $p_i$ são divididos por 5;

(c) esboçar a resposta ao degrau unitário quando os $p_i$ são multiplicados por 5;

(d) comentar os resultados acima;

(e) esboçar a resposta ao degrau unitário quando dois $p_i$ são multiplicados por 5 e o outro é dividido por 5;

(f) comentar os resultados acima.

\end{document}