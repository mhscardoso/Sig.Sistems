\documentclass{article}

% Pacotes extras necessários
\usepackage{amsmath}
\usepackage[lmargin=0.7in, rmargin=0.7in, tmargin=0.5in, bmargin=0.5in, includehead, includefoot]{geometry}
\usepackage{amsfonts}
\usepackage[utf8]{inputenc}
\usepackage[portuguese]{babel}
\usepackage{fancyhdr}

% Sets para outras partes
\setlength{\parindent}{0pt}
\DeclareMathOperator{\sen}{sen}

% ------- Estilo do trabalho -------- %
\fancypagestyle{capa}{
    \fancyhf{}
    \renewcommand\headrulewidth{0pt}
    \fancyfoot[C]{
        Rio de Janeiro\\
        2022
    }
}

\pagestyle{fancy}
\fancyhead{}
\fancyhead[L]{\thepage}
\fancyfoot{}
% ----------------------------------- %

% Dados do Grupo
\title{Sinais e Sistemas - Trabalho 3 - Avaliação 5}
\author{
    \textbf{Grupo 2}\\
    Leonardo Soares da Costa Tanaka\\
    Matheus Henrique Sant Anna Cardoso\\
    Theo Rudra Macedo e Silva
}
\date{}


\begin{document}

% Capa
\maketitle
\thispagestyle{capa}

\newpage

\textbf{1.)} Um SLIT é modelado por $\tau \dot{y}(t) + y(t) = u(t)$ com $y(0^{-}) = \alpha$.
\textbf{G2:} $\tau = 3, \alpha = -2$

(a) Calcular a resposta ao degrau unitário e esboçar o seu gráfico;

(b) calcular a resposta à rampa unitária e esboçar o seu gráfico;

(c) calcular a resposta ao seno $u(t) = \sen(\omega t)$ para $\alpha = 0,\, \omega = 1/(4\tau)$ e esboçar o seu gráfico;

(d) calcular a resposta ao seno $u(t) = \sen(\omega t)$ para $\alpha = 0,\, \omega = 4/\tau$ e esboçar o seu gráfico;

(e) encontrar a entrada $u(t)$ para que $y(t) = \alpha\,\forall t \geq 0$;

(f) encontrar a entrada $u(t)$ para que  $y(t) = 0\,\forall t > 0$;.


\vspace{\baselineskip}


\textbf{2.)} Um SLIT relaxado é descrito por $\ddot{y}(t) + a_1\dot{y}(t) + a_0y(t) = b_1\dot{u}(t) + u(t)$.
\textbf{G2:} $a_1 = 30, a_0 = 3$

(a) Para $b_1 = 0$ calcular a resposta ao degrau unitário e esboçar o seu gráfico;

(b) para $b_1 = 0$ calcular a resposta à rampa unitária e esboçar o seu gráfico;

(c) para $b_1 = 1$ calcular a resposta ao degrau unitário e esboçar o seu gráfico;

(d) para $b_1 = -1$ calcular a resposta ao degrau unitário e esboçar o seu gráfico;

(e) traçar com precisão um único gráfico com as 3 respostas ao degrau;

(f) calcular a resposta a seno $u(t) = \sen(\omega t)$ para $b_1 = 0, \omega = 4a_0$ e esboçar o seu gráfico.


\vspace{\baselineskip}

\textbf{3.)} Um SLIT relaxado é descrito por uma função de transferência com numerador e denominador dados por $n(s) = K$ e $d(s) = (s + p_1)(s + p_2)(s + p_3)$.
\textbf{G2:} $p_1 = 1, p_2 = 3, p_3 = 4$

(a) Esboçar a resposta ao degrau unitário (escolha o valor de $K$);

(b) esboçar a resposta ao degrau unitário quando os $p_i$ são divididos por 5;

(c) esboçar a resposta ao degrau unitário quando os $p_i$ são multiplicados por 5;

(d) comentar os resultados acima;

(e) esboçar a resposta ao degrau unitário quando dois $p_i$ são multiplicados por 5 e o outro é dividido por 5;

(f) comentar os resultados acima.

\end{document}