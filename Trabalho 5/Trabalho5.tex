\documentclass[10pt]{article}

% Pacotes extras necessários
\usepackage{amsmath}
\usepackage[lmargin=0.5in, rmargin=0.5in, tmargin=0.5in, bmargin=0.5in, includehead, includefoot]{geometry}
\usepackage{amsfonts}
\usepackage[utf8]{inputenc}
\usepackage[portuguese]{babel}
\usepackage{graphicx}
\usepackage{fancyhdr}
\usepackage{setspace}

\graphicspath{ {./images/} }

% Sets para outras partes
\setlength{\parindent}{0pt}
\setstretch{1.5}
\DeclareMathOperator{\sen}{sen}

%% Facilidades
%% -- Laplace
\newcommand{\Lap}[1]{\mathcal{L}\left\{#1\right\}}

%% -- Fourier
\newcommand{\Fou}[1]{\mathcal{F}\left\{#1\right\}}

%% -- Transformada Z
\newcommand{\Z}[1]{\mathcal{Z}\left\{#1\right\}}

%% -- Negrito em matemáticas
\newcommand{\bm}[1]{\boldsymbol{#1}}


% ------- Estilo do trabalho -------- %
\fancypagestyle{capa}{
    \fancyhf{}
    \renewcommand\headrulewidth{0pt}
    \fancyfoot[C]{
        Rio de Janeiro\\
        2022
    }
}

\pagestyle{fancy}
\fancyhead{}
\fancyhead[L]{\thepage}
\fancyfoot{}
% ----------------------------------- %

% Dados do Grupo
\title{Sinais e Sistemas - Trabalho 5 - Avaliação 9}
\author{
    \textbf{Grupo 2}\\
    Leonardo Soares da Costa Tanaka\\
    Matheus Henrique Sant Anna Cardoso\\
    Theo Rudra Macedo e Silva
}
\date{}

\begin{document}
\maketitle
\thispagestyle{capa}
\newpage

%% Questão 1 -----------------
\textbf{1.) Um SLIT relaxado é descrito pela equação a diferenças $y_{k + 2} + \alpha y_k = \beta u_{k + 1} + \gamma u_k$.\\
Os dados $(\alpha, \beta, \gamma)$ são: \textbf{G2: }$(-1/4, 1, 2)$;}

Utilizando as constantes do grupo, a equação torna-se: $y_{k + 2} - \frac{1}{4} y_k = u_{k + 1} + 2 u_k$

(a) Encontrar a função de transferência $G(z)$;

(b) encontrar polos e zeros e verificar a estabilidade;

(c) encontrar as equações dinâmicas;

(d) calcular, iterativamente, os 10 primeiros valores $y_k$ para entrada em degrau unitário;

(e) usando transformada em $Z$, encontrar uma expressão analítica para a $y_k$;

(f) comparar os resultados iterativo e analítico.


\vspace{\baselineskip}


%% Questão 2 ------------------
\textbf{G2: 2.) Eis um ``Problema de Algibeira": um vendedor de queijos efetua apenas transações do tipo \emph{vende metade de seu estoque mais meia peça}. Pede-se o número inicial de peças, $x_0$ se após a 6ª venda seus queijos acabam. Interpretar a filosofia de vendas por meio de uma equação a diferenças e calcular $x_k$, o saldo de estoque após a $k$-ésima venda. Dar a resposta ao problema de algibeira.}


\vspace{\baselineskip}

%% Questão 3 --------------------
\textbf{3.) Para a EDLIT $\tau \dot{y}(t) + y(t) = \beta u(t)$ com $y(0^-) = y_0$:\\
\textbf{G2: } $\tau = 2, \beta = 2 \text{ e } y_0 = 1$.}

Utilizando as constantes do grupo, a equação torna-se: $2 \dot{y}(t) + y(t) = 2 u(t)$ com $y(0^-) = 1$

(a) Calcular a resposta à rampa unitária por Laplace, e traçar com precisão o seu gráfico para $t \in [0,\,5\tau]$;

(b) por Euler I, encontrar a equação que relaciona $y_k \leftrightarrow y(kT)$ e $u_k \leftrightarrow u(kT)$;

(c) resolvê-la por transformada $Z$ para entrada em rampa;

(d) listar as sequências obtidas para $T = \tau, T = \tau / 2, T = \tau / 4 \text{ e } T = \tau / 8$;

(e) plotar os valores de $y(kT)$ no mesmo gráfico do ítem (a) e comparar as aproximações numéricas;

(f) repetir (b), (c), (d) e (e) para Newton.

\vspace{\baselineskip}


%% Questão 4 ----------------------------
\textbf{4.) Para a EDLIT $\ddot{y}(t) + 2\zeta \omega_n \dot{y}(t) = \alpha \dot{u}(t) + \beta u(t)$ com CIs nulas:\\
Os dados $(\zeta,\, \omega_n,\, \beta,\, \alpha)$ são \textbf{G2: }$(1/3,\, 2,\, 3,\, -1)$.}

Utilizando os dados do grupo, temos: $\ddot{y}(t) + \frac{4}{3} \dot{y}(t) = -\dot{u}(t) + 3u(t)$

(a) para $u(t) = 1$, encontre, por Laplace, a solução e plote-a com precisão para $t \in [0,\, 8/(\zeta \omega_n)]$;

(b) por meio de variáveis $x_1$ e $x_2$ apropriadas, expressá-la como $\dot{\bm{x}}(t) = A\bm{x}(t) + B\bm{u}(t)$ e $\bm{y}(t) = C\bm{x}(t) + D\bm{u}(t)$;

(c) por Euler I, relacione $\bm{x}_k \leftrightarrow \bm{x}(kT), \, y_k \leftrightarrow y(kT)$ e $u_k \leftrightarrow u(kT)$ (o procedimento para vetores é o mesmo e leva a $\bm{x}_k = A_d \bm{x}_{k - 1} + B_d \bm{u}_k$ e $\dot{\bm{y}}_k = C_d \bm{x}_k + D_d \bm{u}_k$) e obtenha $y_k$;

(d) plota as sequências obtidas para $T = T_0 = 1 / (\zeta \omega_n), \, T = T_0/2, \, T = T_0/4$ e $T = T_0/8$ no gráfico de (a) e compare as aproximações numéricas.


\vspace{\baselineskip}

%% Questão 5 ------------------------------
\textbf{5.) Seja EDVT $\dot{y}(t) + \alpha(t)y(t) = u(t)$ com $u(t) = 1(t)$. Quando um sinal contínuo $p$ tende a uma constante para valores altos de $t$ ($\lim p(t) = p_r = $ cte. para $t \to \infty$) esta é chamada de valor de regime do sinal.}

(a) sem resolver a equação calcular o valor de regime $y_r$, supondo que $y(t)$ tende a ele;

(b) por Euler I, relacione as sequências $u_k \leftrightarrow u(kT)$ e $y_k \leftrightarrow y(kT)$;

(c) resolva, manualmente ou por meio de um script em alguma linguagem, para $T = 1, T = 1/2, T = 1/4$ e $T = 1/10$; a solução $y(t)$ deve ser aproximada para $t \in [0, \, 10]$;

(d) repetir (b) e (c) para Euler II.

Os dados $a(t)$a e $y(0^-)$ são: \textbf{G2: }$(t^2 - 1)/(t^2 + 1)$ e $-1$.

\end{document}
